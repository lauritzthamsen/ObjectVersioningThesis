\chapter{Introduction} \label{chapter:INTRODUCTION}

% Background
Programming systems such as Squeak/Smalltalk~\cite{Ingalls1997Squeak,GoldbergRobson83} and REPLs for LISP or Python allow adapting programs at runtime.
Changes to programs in such environments are effective immediately and programmers can see or test right away what differences their actions make.
Thus, these systems provide immediate feedback to programmers.\\
A subset of such systems, which includes, for example, Self~\cite{Ungar1987SPS,Ungar2007SEL} and the Lively Kernel~\cite{Ingalls2008LKS,Krahn2009LWD}, are those built around prototype-based object-oriented languages~\cite{Lieberman1986UPO}.
In prototype-based systems programmers create applications using objects and without having to define classes first.
In Self and the Lively Kernel, programmers can inspect and change the state and behavior of objects at runtime.
Programmers create actual objects, not source code that only abstractly describes potential objects.

The Lively Kernel was designed to support this kind of development~\cite{Lincke2013UTW}.
It provides tools to directly manipulate the style, composition, and scripts of graphical objects.
For example, programmers can change the positions and composition of objects directly using the mouse.
They can use temporary workspaces to manipulate objects programmatically.
They can edit and try methods directly in the context of graphical objects.\\
For example, to add new functionality to a graphical application, a Lively Kernel user might copy an existing button object and then modify the new button object: move the new button to a sensible position, resize it, set a new label, and add a script to be executed on mouse clicks.
The user makes all changes directly to one button object.
How this button fits into the application's interface is visible at all time.
Clicking the button allows to directly test its functionality.
This way, the Lively Kernel allows for fast feedback, especially during the development of graphical applications.

% Problem
Programmers' changes to objects can turn out to be inappropriate.
Programmers can, for example, accidentally change positions or connect the wrong objects when manipulating applications with mouse interactions.
They might try a couple of different alternatives such as different colors and layouts, only to realize that an earlier state was most appealing.
Similarly, programmers might learn in hindsight that making a change to an object's scripts introduced an error or impacts the application's performance.
They might make a mistake in a code snippet, which then manipulates many objects.
Moreover, problematic changes can be introduced when code is evaluated only to understand or test behavior, not to permanently change state.

% So what?
However, when changes turn out to be problematic, programmers often need to undo the changes manually.
The Lively Kernel does not provide an undo for changes to objects.
This is especially at odds with the Lively Kernel's support for trying ideas right away: Developers are able to make changes directly and receive immediate feedback, but do not get support when such changes turn out to be inappropriate.
Thus, to recover a previous development state, programmers often need to manually reset the state to how it previously was---probably using the same tools the changes were initially made with.
Furthermore, this potentially involves multiple properties of multiple objects changed by multiple developer actions.\\
The Lively Kernel provides tools to commit and load versions of objects.
In case such commits exist, programmers can load earlier versions of objects to re-establish previous states.
Nevertheless, depending on how far the latest version is from the actually desired state, manual changes might still be necessary.
To keep the effort to re-establish \emph{any} previous state low, programmers would need to commit many versions.
However, this contradicts the goal as commiting many versions is also a significant effort.
Some commits would be made only to protect intermediate states, not to share and document results.
Especially when the preserved versions should be usable and documented, programmers would be required to test and describe many versions.\\
\emph{In summary}, recovering previous states of objects in the Lively Kernel is currently a significant effort for programmers.
They either have to manually re-set changed state or need to take time-consuming precautionary actions.

A typical approach to implementing multi-level undo for the changes to application state is the Command pattern~\cite{GammaHelmJohnsonVlissides95}.
The Command pattern packages changes into actions.
These actions can then be recorded to be able to subsequentely undo them.
This requires developers to implement undos for all possible actions.
Therefore, an implementation of the command pattern---even when limited to the Lively Kernel tools that manipulate objects---would be rather comprehensive.
Using the Command pattern would also require developers to follow the pattern when implementing new tools.
Furthermore, the Command pattern is entirely unpractical for undoing the effects of evaluating arbitrary code from the Lively Kernel's workspaces and editors.

Worlds~\cite{Warth2011Wor,Warth2009PhD}, in contrast, is a more generic approach for controlling the scope of side effects.
Code is executed in \emph{world} objects, which capture all side effects.
The worlds can then be used to run code with particular sets of changes.
Developers could create new worlds for all their actions and discard worlds to return to previous states when necessary.
Therefore, it still requires programmers to explicitly take precautionary actions, similar to version control systems.
In addition, the implementation of Worlds in JavaScript is not yet practical.
For example, it currently prevents garbage collection.

CoExist~\cite{Steinert2012COE,Steinert2014EVA} provides automatic recovery support without requiring developers to take precautionary actions.
CoExist automatically records versions for every change and, thereby, provides a fine-grained history of intermediate development states.
Programmers can review the changes chronological, examine the impact each change had, and re-establish previous versions.
However, CoExist currently recognizes only changes made to the source code of classes.
Its versions do not include the state of objects.

% Solution
This thesis proposes an approach for versioning the entire state of programming systems as basis for automatic recovery support.
In particular, this thesis introduces an approach to preserving and managing versions of all objects using alternative, \emph{version-aware} references.
Version-aware references are alternative references as they refer to multiple versions of objects.
They resolve transparently to particular versions.
Versions of objects are preserved together, so that version-aware references can be resolved transitively to the state of a particular moment.
For this to be practical, versions of objects are kept in the application memory and the state of all versions is preserved incrementally on writes.
To which versions the version-aware references resolve can be changed without significantly interrupting program execution:
The version-aware references select the current versions dynamically instead of being hard-wired to specific versions.

We implemented our approach in JavaScript.
The implementation does not require adaptions to established execution engines.
Proxies are used to implement version-aware references: conventional references point to the proxies and the proxies delegate all object interactions transparently to particular versions of objects.
Source transformations introduce proxies consistently for all objects.
Therefore, programmers do not need to adapt their programs manually.

The approach supports fine-grained histories of development states.
Not every state can be re-established, but versions that have been preserved.
In this, the presented solution is a basis for recovery support that continuously preserves versions.

\section{Contributions}

The goal of this work is to provide object versioning for the Lively Kernel.
To that effect, the main contributions of this thesis are the following:

\begin{itemize}
    \item An approach to object versioning for systems like the Lively Kernel based on version-aware references that transparently delegate to one of multiple versions of an object (Section~\ref{sec:APPROACH:1}).
    \item A design that provides the proposed version-aware references through proxies for the Lively Kernel (Section~\ref{sec:APPROACH:2}).
    \item An implementation of the design in JavaScript that can be used to effectively preserve and re-establish development states of the Lively Kernel (Section~\ref{chapter:IMPLEMENTATION}).
\end{itemize}


\section{Thesis Structure}

The remainder of this thesis is organized as follows. 
Chapter~\ref{chapter:BACKGROUND} describes prototype-based programming systems, CoExist, and the Lively Kernel.
Chapter~\ref{chapter:MOTIVATION} illustrates how developers directly manipulate objects in the Lively Kernel and exemplifies recovery needs.
Chapter~\ref{chapter:APPROACH} introduces our approach to object versioning and describes how proxies can be used for concrete solutions.
Chapter~\ref{chapter:IMPLEMENTATION} presents our implementation for the Lively Kernel, which Chapter~\ref{chapter:EVALUATION} then evaluates in terms of functionality and practicability.
Chapter~\ref{chapter:RELATED_WORK} compares our solution to related work.
Chapter~\ref{chapter:FUTURE_WORK} presents future work, while Chapter~\ref{chapter:SUMMARY} concludes this thesis.
