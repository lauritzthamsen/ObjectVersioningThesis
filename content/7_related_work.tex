\chapter{Related Work} \label{chapter:RELATED_WORK}

This chapter presents two categories of related work:

\begin{enumerate}
    \item Approaches related to our motivation and, thus, to providing access to previous versions of the system state.
    \item Approaches related to our technical solution and, thus, to combining changes into first-class objects that can be used to scope changes.
\end{enumerate}


\section{Recovering Previous System States}

The approaches presented in this section support programmers in recovering previous states without requiring programmers to create snapshots in advance.


\subsection{CoExist}

CoExist~\cite{Steinert2012COE} provides recovery support through continuous versioning in Squeak/Smalltalk.
For each change made to source code, CoExist creates a new version of the system sources, resulting in a fine-grained history of changes.
CoExist presents this history in a timeline tool and a dedicated browser.
For each version, those tools show the changes, test results, and a screenshot.
Developers can recover previous development states, even without taking precautionary actions beforehand.
This way, developers can concentrate on implementing their ideas and let CoExist record the required versions to be able to recover when necessary.

Both CoExist and our approach to object versioning allow multiple versions of the development state to coexist.
With both approaches, preserved versions are part of the program runtime and can be re-established easily.
Currently only CoExist records versions continuously on the granularity of changes made by developers.
CoExist provides much more tool support to find and recover changes from previous versions.
However, CoExist recognizes only changes to the source code of classes, while our system preserves the state and behavior of objects.


\subsection{Lively Kernel Offline Worlds}

Offline Worlds~\cite{Czuchra2012OfW} is an approach to protect state against system failures by saving it periodically.
In a fixed time interval the current Lively Kernel world is saved automatically as protection against unexpected crashes and network outages.
The implementation serializes the state and object-specific behavior of all morphs.
For each serialized state only the differences to the previous state is stored.
Further, the implementation uses client-side storage for fast access and to safeguard against outages of both the client-side and the server-side.
As only the differences to the last saved version is stored and previous versions, therefore, remain available, Offline Worlds could, conceivably, also be used to re-establish other versions than the latest, but does not provide support for this.

Offline Worlds preserves the latest state of a Lively Kernel world to support recovery from system failures, while our approach preserves multiple versions of the runtime to provide recovery when programmers make inappropriate changes.
Offline Worlds only preserves the state of all graphical objects.
In contrast, our system also recognizes changes to classes, globally accessible state, and morph state explicitly excluded from serialization.
Further, our approach saves versions incrementally and re-establishes versions dynamically, while Offline Worlds saves and loads versions of the world in discrete, interruptive steps.
Even when not used to recover from a system failure, Offline Worlds still requires to re-load the entire world, while our system only switches which versions of particular objects should be used and even preserves object identity of these through the version-aware references.


\subsection{Back-in-Time Debugging}

Back-in-time Debuggers~\cite{Lewis2003BIT}, also known as \emph{Omniscient Debuggers}, allow developers to inspect previous program states and step backwards in the control flow to undo the side effects of statements.
Approaches for this are either based on logging or replay: either the debugger records information to be able to recreate particular previous situations, requiring mainly space for the different states, or the debugger re-executes the program up to a particular previous situation, requiring mainly time to re-run the program.
While many logging-based approaches introduce significant execution overheads, replay-based approaches have to ensure that the program is re-executed deterministically, which can be a difficult problem when, for example, programs can rely on state outside of the program runtime such as the content of files or the state of other programs.

Our approach is more related to logging-based back-in-time debugging.
It allows re-establishing a previous state through preserving information.
However, back-in-time debuggers need to be able to undo the effects of each statement separately, while our system's versioning granularity is arbitrarily and can, for example, correspond to programmer interactions with the system.
In general, back-in-time debuggers support a particular development task---debugging---and, thus, are also often only active when a program is started in a separate debugging mode.
In contrast, the purpose of object versioning is more comprehensive.
We expect object versioning to be active at least during all development tasks, but possibly even be enabled at all times.


\subsection{Software Transactional Memory}

\ac{stm}~\cite{Shavit1995STM} captures changes to values in transactions, analogous to database transactions.
Each transaction has its own view of the memory, which is unaffected by other concurrently running transactions.
Multiple versions of the system state can coexist and which version is read and written to depends on the transaction.
Transactions contain a number of program statements that are executed atomically.
The changes from a transactions are only permanent when no conflicts occur with other transactions.
On conflicts, all changes from the transaction are rolled back and undone.

\ac{stm} and our approach are similar in that multiple versions of the system state can coexist and that a previous state can be re-established if necessary.
However, \ac{stm} provides concurrency control and an alternative to lock-based synchronization, while our approach provides recovery support to developers when changes turn out be inappropriate.
\ac{stm} transactions are automatically rolled back when changes conflict with changes from other concurrently running transactions, while our versions are offered to programmers to undo changes when necessary.
Programmers can actively decide to undo changes when these, for example, negatively impact the functionality, design, or performance of programs. 
Our versions of the runtime are also first-class objects, which can be stored in variables and be re-established at any time, while transactions are always created implicitly through particular control structures and commited immediately upon success.



\section{Dynamically Scoping First-class Groups of Changes}

The approaches presented in this sections allow to combine changes into first-class objects and run code with particular sets of changes.


\subsection{Worlds}

Worlds provide a language construct for controlling the scope of side effects: changes to the state of objects are by default only effective in the \emph{world} in which the changes occurred.
These worlds are first-class values and can be used to execute statements with particular side effects being active.
A new world can be spawned from an existing world, which establishes a child-parent relationship between the two worlds.
Developers can commit changes from a child world to its parent world, thereby extending the scope of the captured side effects.
The Worlds approach includes conditions that prevent commits that would potentially introduce inconsistencies.

In comparison, Worlds provides a language construct for experimenting with different states of the system, while object versioning allows to preserve versions of the system to recover previous states: Our approach does not include extensions to the host programming languages and no conditions for combining versions with their predecessor versions, but provides a basis for CoExist-like continuous versioning and recovery tools.

Other differences between Worlds and our approach regard the implementations.
Our implementation in JavaScript does not prevent garbage collection as Worlds does.
Further, both use different libraries for source transformations.
Our source transformations are faster and do not use JavaScript exceptions.


\subsection{Object Graph Versioning}

Object Graph Versioning\cite{Pluquet2009ECP} allows programmers to preserve access to previous states of objects.
Fields of objects can be marked as \emph{selected} fields.
When a snapshot is created, the values of these selected fields are preserved.
Therefore, not every state can be re-established, but states that are part of global snapshots. 
The approach, thus, provides fine-grained control to programmers regarding which fields of which objects should be preserved when.

The technical solution is similar to our design.
Analogous to our proxy-based version-aware references, selected fields do not refer directly to their actual values, but to chained arrays that manage multiple versions of the state of a field and delegate access to the current version transparently.
The chained arrays decide which version to retrieve and when to create new versions using a global version identifier.
In constrast to our sulution, individual fields are versioned and only when programmers explicitly mark them as selected.

Object Graph Versioning aims to support implementing application-specific undo/redo or tools like back-in-time debuggers.
In contrast, our approach to object versioning aims to support recovery of previous system states during the development of arbitrary applications.


\subsection{Context-oriented Programming}

\ac{cop}~\cite{Hirschfeld2008COP,Appeltauer2009CCP} adds dedicated language constructs for dynamic behavior variations.
Depending on context information, \ac{cop} allows to enable and disable layers, which contain methods to be executed instead or around methods of the base programs.
Context information can be any information which is computationally accessible.
Layers can be enabled and disabled at runtime.
Different implementations of \ac{cop} provide different mechanisms to scope the activation of layers: for example, layers can be activated explicitly for a particular scope or globally for the entire runtime.
ContextJS~\cite{Lincke2011OIC} it is possible to activate layers for specific objects.

In comparison to our approach, \ac{cop} allows to activate combinations of layers, while our system executes code using a single active version.
In \ac{cop} layers are indepedent, while our versions are predecessors and successors of each other.

\ac{cop} aims at supporting the separation of heterogeneous cross-cutting concerns, while object versioning aims at supporting developers with the recovery of previous states.
However, \cite{Lincke2012SCS} showed that \ac{cop} can also be used to experiment with changes to a system: developers can implement experimental changes to behavior in layers, not to modularize context-dependent adaptions, but to be able to scope changes dynamically and recover the original system behavior easily.
However, this requires programmers to make experiments explicitly.
They need to use layers for their adaptions, enable the layers for test runs, and move code from layers back to the base system when experiments are successes and they want to maintain the original modularization of the system.
\ac{cop} also allows only behavior variations, while our approach recognizes changes to both state and behavior.


\subsection{ChangeBoxes}

ChangeBoxes~\cite{Denker2007EEC} is an approach to capturing and scoping changes to a system using first-class entities, called ChangeBoxes.
A ChangeBox can contain changes to multiple elements of a software system such as adding a field, removing a method, or renaming a class.
The approach does not constrain how changes get grouped into ChangeBoxes, but every change has to be encapsulated by a ChangeBox.
Each ChangeBox can be used for setting the set of active changes for the scope of a running process.
This way, multiple running processes can view the system differently by using different ChangeBoxes.
ChangeBoxes can have ancestor relations and merge changes from multiple ancestors.
With the ancestor relations, ChangeBoxes can be used to review the evolution of systems and to undo changes.

The ChangeBoxes approach is similar to our approach as changes to the system are grouped into first-class objects and these can be used to run code in different versions of the runtime.
Furthermore, with both solutions there is no definite notion in how changes are grouped into versions.
Our object versioning approach is intended to be used to group changes associated with developer actions and a simple global undo/redo mechanism to undo inappropriate actions is built into our solution.
To actually undo changes ChangeBoxes, in contrast, is rather tedious~\cite{Steinert2012COE}.
Moreover, ChangeBoxes recognizes only changes to the static elements of a software system such as packages, the structure of classes, and methods.
Object versioning, in contrast, preserves the state and behavior of objects.


\subsection{Practical Object-oriented Back-in-Time Debugging}

Practical Object-oriented Back-in-Time Debugging~\cite{Lienhard2008POB,Lienhard2008PhD} is a logging-based approach to back-in-time debugging that uses alternative references to preserve the history of objects.
These alternative references, called \emph{Aliases}, are actually objects and part of the application memory.
These objects contain information about the history and origin of the values stored in fields.
Aliases are not passed around, but instead are created for each read or write of a field and for each value passed as parameter.
Each alias refers to an actual value, but also to another alias---its \emph{predecessor}---representing the value previously stored by a field and to the alias that was used to create this new alias from---its \emph{origin}.
An alias and its origin both refer to the same value, but provide different information on their creation context, which is a particular method.
The origin link can be used to follow the object's ``flow'' through the program.
Each alias also records a timestamp on its creation and with this information the predecessor link can be followed to read a value as it was at a particular moment in time.

In comparison, with aliases it is possible to recreate all states the system was in and also retrace the flow of all values, while our system stores only particular versions.
Such versions could, for example, correspond to programmer interactions, so that programmers can undo the effects of particular actions easily.
Another difference between object versioning with version-aware references and reverse engineering with aliases is the existence of modes.The alias references are intended to be used in explicit debugging sessions, while our version-aware reference are intended to be used at all times.
