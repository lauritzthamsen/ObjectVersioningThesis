\begin{abstract}

% Background
In programming systems such as the Lively Kernel, programmers construct applications from objects.
Dedicated tools allow them to manipulate the state and behavior of objects at runtime.
Programmers are encouraged to make changes directly and receive immediate feedback on their actions.

% Problem Statement
When programmers, however, make mistakes in such programming systems, they need to undo the effects of their actions.
Programmers either have to edit objects manually or re-load parts of their applications.
Moreover, changes can spread across many objects.
As a result, recovering previous states is often error-prone and time-consuming.

% Approach
This thesis introduces an approach to object versioning for systems like the Lively Kernel.
Access to previous versions of objects is preserved using \emph{version-aware references}.
These references can be resolved to multiple versions of objects and, thereby, allow re-establishing preserved states of the system.

% Results
This thesis presents a design based on proxies and an implementation in JavaScript.
The evaluation of this implementation shows that the Lively Kernel can run with our version-aware references and that preserved system states can be re-established.
While the memory overhead of the version-aware references is reasonable, the execution overhead is not yet practical.
However, with performance improvements, the solution could be used to provide practical recovery support to programmers.

\end{abstract}



\begin{zusammenfassung}

{\Large Objekt Versionierung für den Lively Kernel: Erhaltung vorheriger Systemzustände in einem objekt-orientierten Programmiersystem}

In Programmiersystemen wie dem Lively Kernel können Programmierer Anwendungen aus Objekten erstellen.
Dabei erlauben dedizierte Werkzeuge den Zustand und das Verhalten von Objekten zur Laufzeit zu verändern.
Programmierer werden ermutigt Änderungen direkt zu machen und erhalten umgehend Feedback.

Wenn Programmierer in solchen Programmiersystemen jedoch Fehler machen, müssen sie Änderungen rückgängig machen.
Dazu müssen sie entweder Objekte manuell bearbeiten oder Teile ihrer Anwendungen neu laden.
Die gemachten Änderungen können dabei über viele Objekte verteilt sein.
Vorherige Zustände wiederherzustellen ist deshalb häufig schwierig und zeitaufwendig.

Diese Arbeit stellt einen Ansatz für die Versionierung von Objekten in Systemen wie dem Lively Kernel vor.
Der Ansatz basiert auf \emph{versionsbewusste Referenzen}.
Diese können zu mehreren Versionen von Objekten aufgelöst werden und erlauben so vorherige Systemzustände wiederherzustellen.

Die Arbeit beschreibt einen auf Proxys basierenden Entwurf und eine Implementierung in JavaScript.
Die Evaluierung der Implementierung zeigt, dass Systemzustände des Lively Kernels damit erhalten und wiederhergestellt werden können.
Der zusätzlich nötige Arbeitsspeicher für die versionsbewussten Referenzen ist dabei vertretbar, während die Programmausführung erheblich verlangsamt wird.
Mit Verbesserungen könnte die vorgestellte Lösung allerdings benutzt werden, um Entwickler mit praktikablen Wiederherstellungs-Werkzeugen zu unterstützen.

\end{zusammenfassung}
