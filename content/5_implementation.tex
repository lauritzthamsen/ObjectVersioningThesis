% Proxies stand-in for multiple versions, intercept all access, and forward to the correct version.


\chapter{Implementation} \label{chapter:IMPLEMENTATION}

This chapter describes how we used the ECMAScript 6 proxies to implement version-aware references for the Lively Kernel.
It presents the proxy's behavior and shows how proxies are inserted for ordinary references using source transformations.
The chapter also presents implemented workarounds for the current state of ECMAScript 6 proxies.
It concludes with current limitations.


\section{ECMAScript 6 Proxies as Version-aware References} \label{sec:IMPLEMENTATION:1}

We implemented version-aware references in JavaScript with the proxies proposed with ECMAScript 6\footnote{\url{http://wiki.ecmascript.org/doku.php?id=harmony:direct_proxies}, accessed February 3rd, 2014}, the next version of JavaScript's standard.

\subsection{ECMAScript 6 Proxies}

% TODO:    
    % intro: stand-in for objects and intercept all kinds of object interactions, e.g. a proxy could be used to log all access to the properties of an object: a client object has access to a server object only via the proxy, the proxy intercepts the reads and writes and forwards them only after logging
    
    % setup: new Proxy (code snippet) -> diagram (client, proxy, proxy handler, target), traps (logging: get + set traps)
    
    % examples of other interactions that can be trapped (functions -> apply + construct, enumeration, built-in functions of \lstinline{Object}..), list of all traps a proxy can have
    
    % subsubsection: as virtual objects


\iffalse
\begin{verbatim}\fi
\begin{code}{}{}
var proxy = new Proxy(target, proxyHandler);
\end{code}
\label{lst:example}
\iffalse
\end{verbatim}\fi

The behavior of the proxies is specified by a \emph{Handler}, a JavaScript object that is supplied when the proxy is created.
The handler can implement specific methods called \emph{Traps}.


\iffalse
\begin{verbatim}\fi
\begin{code}{}{float,numbers=left}
var client = {},
    server = {openSecret: "I don't like Mondays"},
    proxyHandler = {
        get: function(target, name) {
            console.log(name + ' was read at ' + Date());
            
            return target[name];
        }
    }

client.server = new Proxy(server, proxyHandler);
\end{code}
\label{lst:loggingProxy}
\iffalse
\end{verbatim}\fi


\begin{figure}[h]
    \centering
    \includegraphics[width=\textwidth]{figures/5_implementation/1_loggingProxy.pdf}
    \caption{TODO}
    \label{fig:LoggingProxy}
\end{figure}


\lstinline{client.server.openSecret}: ``openSecret was read at Sat May 10 2014 23:00:54 GMT+0200 (CEST)''




% Table~\ref{table:traps} lists all implemented proxy traps.
% 
% \begin{table}[h]
% \begin{center}
% \begin{tabular}{|l|l|r|}
% \hline
% get: function(target, name, receiver) \\ \hline
% set: function(target, name, value, receiver) \\ \hline
% apply: function(target, suppliedThisArg, suppliedArgs) \\ \hline
% construct: function(target, args) \\ \hline
% has: function(target, name) \\ \hline
% hasOwn: function(target, name) \\ \hline
% defineProperty: function(target, name, desc) \\ \hline
% deleteProperty: function(target, name) \\ \hline
% getOwnPropertyDescriptor: function(target,name) \\ \hline
% getOwnPropertyNames: function(target) \\ \hline
% getPrototypeOf: function(target) \\ \hline
% freeze: function(target) \\ \hline
% seal: function(target) \\ \hline
% preventExtensions: function(target) \\ \hline
% isFrozen: function(target) \\ \hline
% isSealed: function(target) \\ \hline
% isExtensible: function(target) \\ \hline
% enumerate: function(target) \\ \hline
% keys: function(target) \\ \hline
% \end{tabular}
% \caption[Table caption text]{Traps implemented by our proxy handlers.}
% \label{table:traps}
% \end{center}
% \end{table}

The traps fire either when the proxy is accessed with specific JavaScript operators or when it is passed to meta-programming facilities provided by built-in globals\footnote{\url{http://wiki.ecmascript.org/doku.php?id=harmony:direct_proxies\#api}, accessed March 5, 2014}.
For example, the \lstinline{preventExtensions}-trap fires when a proxy is passed to the built-in \lstinline{Object.preventExtensions()} function, which prevents subsequentely adding new properties to an object.








\subsection{Versioning Proxies}


\begin{figure}[h]
    \centering
    \includegraphics[width=\textwidth]{figures/5_implementation/2_versioningProxy.pdf}
    \caption{TODO}
    \label{fig:LoggingProxy}
\end{figure}


% TODO:
    % we call our usage/configuration of proxies \emph{versioning proxies}

    % proxies as stand-ins for multiple versions of objects: no fixed target but versions dictionary (diagram)
    % only the proxy handler has access to the versions. therefore, the garbage collector reclaims proxies and versions together
    
    % versioning proxy handler: traps forward to the current version, conceptual code of the get trap as example
    
    % selecting the right version: currentVersion(), version of the system (objects with id, predecessor, successor), current system version globally accessible via lively.CurrentVersion, used to retrieve the correct version from the version dictionary
    % global version identifier is okay: single-threaded execution + cooperative scheduling
    
    % new versions of the system: object versions of previous system versions are not allowed to change. write access / manipulating object interactions is forwarded to object versions of the current system version. when no such object versions exist, new version are created in the traps that intercept the manipulations (copy-on-write in set+seal...-traps). full copy of objects and inserted as new version into the version dictionary

    % subsection: scope of the versioning, idea: all mutable objects to be able to undo the changes to all objects. however, some objects cannot be versioned (DOM objects).
    % obviously also excludes all state external to the JavaScript runtime.




no fixed target

Our proxy handlers have a reference to the versions of the object the proxy represents.
 a JavaScript object to map version identifiers to the actual versions of the object.

Figure~\ref{fig:VersioningProxy} shows the relationship between a proxy, the proxy handler, and the versions of an object.

Having the handler object hold all the versions of conceptionally one object takes care of garbage collection.
When a proxy gets garbage collected, the garbage collector also reclaims all versions.


Our proxy handler uses the traps to forward all interactions to the current version of an object.
Its \emph{get}-trap is implemented similarly to the \lstinline{get} function shown in Listing~\ref{lst:getTrap}.

\iffalse
\begin{verbatim}\fi
\begin{code}{TODO}{float,numbers=left}
get: function(name) {
    var version = this.currentVersion();
    
    return version[name];
}
\end{code}
\label{lst:getTrap}
\iffalse
\end{verbatim}\fi
% 
% First, the current version gets determined using the \lstinline{currentVersion()} helper function.

% In the case that we are reading a property named \lstinline{age} from an object \lstinline{person}, all these conditions fall through.
% The trap reads the property from the current version and returns the result.


% The proxy handler chooses a \emph{version of an object} according to a global \emph{version of the system}.
% The version of the system is an ordinary JavaScript object with three properties: an \lstinline{ID}, a \lstinline{previousVersion}, and a \lstinline{nextVersion}.
% The current version is always referred to by \lstinline{lively.CurrentVersion}.
% It is, therefore, globally accessible.
% The proxy handler uses system version's \lstinline{ID} to look up the version of the object in its version dictionary.





% When a version of an object is not changed in versions of the system, it's latest version continues to reflect the current state.
% Therefore, when a proxy traps read access, it looks up the latest version object from its version dictionary.
% When a proxy, however, traps write access and no version is available for the current system version, a new version of the object is created before forwarding the write access.
% For this, the latest available version shallowly copied and inserted into the version dictionary.
% 
% For most objects, write access is only trapped in the following traps: \emph{set}, \emph{defineProperty}, \emph{deleteProperty}, \emph{preventExtension}, \emph{seal}, and \emph{freeze}.\\
% Arrays, however, have built-in functions such as \lstinline{push} and \lstinline{pop} that also manipulate an array's state.
% Therefore, the \emph{apply}-trap also checks for write access and creates new versions when necessary.
% 
% To change the system state, only the global version identifier needs to be changed.
% Depending on whether the version is set to a previous version, a following version, or a new version, changing the identifier is an undo, a redo, or a commit of the current state.
% 
% Using a global version of the system is reasonable for now as JavaScript is executed single-threaded and scheduled cooperatively by the JavaScript engines.
% That is, while a JavaScript script runs, the global version cannot be changed by another script.




 
\paragraph{Scope of the Versions}
Using the proxies and system versions, the state of all mutable JavaScript objects can be preserved and re-established.
Certain host objects in JavaScript engines, however, cannot be versioned.
For example, the objects that represent the elements of the browser's \ac{dom} cannot be copied.
To still have changes to graphical objects be visible immediately, we update the HTML document from the current set of visible morphs manually when the system version changes.

% 
% % TODO: add this, as we refer to it from the evaluation:
% These objects consume space, but cannot be versioned with our approach.
% They are referred to from the browser's \ac{dom} which is external to the JavaScript runtime and which, thus, does not use version-aware references.
% Moreover, some of these objects cannot be copied to create new versions.













\section{Proxies For All Mutable JavaScript Objects}

To have the proxies intercept access to all mutable state, objects need to be accessed only via proxies.

Objects, arrays, and functions are mutable in JavaScript.
In fact, functions and arrays are also objects.
They can have arbitrary properties.

% TODO: code snippet as example: line 2 of the following returns true:
    % var array = []; array.func = function() {};
    % array.func.prop = 3; array.func.prop === 3 


Accessing all objects only via proxies is achieved by creating proxies right when objects get created and then returning references to the proxies instead of to the objects.
% To achieve this, we create proxies immediately for new objects and return the proxies instead of the objects, so a reference to the proxy gets passed around instead of the reference to the actual object, which instead is captured as a version in the versions dictionary of the proxy.


In JavaScript, there are three different ways to create an object: 
\begin{itemize}
    \item evaluating literal expressions: e.g. \lstinline|{age: 12}|
    \item calling specific built-in functions: e.g. \lstinline|Object.create(prototype, {age: 12})|
    \item using functions as constructors: e.g. \lstinline|new Person(12)|
\end{itemize}



% TODO: summary sentence for the upcoming subsections...
Our implementation uses a combination of source transformations and behavior of proxy traps to return proxies from the expressions that otherwise would return new objects.




\subsection{Transforming Literal Expressions}

We use source transformations to wrap literal expressions consistently into a \emph{proxyFor()} function, which takes an object as parameter and returns a proxy for it.
Therefore, we wrap the literal forms of objects, arrays, and functions.

The transformations of literal objects, arrays, and functions are shown by example in Table~\ref{table:literalTransforms}.

\begin{table}[h]
\begin{center}
\begin{tabular}{| l | l | l |}
\hline
Type & Input & Output \\ \hline
\emph{Objects} & \lstinline|{name: 'James', age: 24}| & \lstinline|proxyFor(name: 'James', age: 24)| \\ \hline
\emph{Arrays} & \lstinline|[person1, person2]| & \lstinline|proxyFor([person1, person2])| \\ \hline
\emph{Functions} & \lstinline|function (a, b) {..}| & \lstinline|proxyFor(function (a, b) {..})| \\ \hline
\end{tabular}
\end{center}
\caption[Table caption text]{Transformations of literal objects, arrays, and functions.}
\label{table:literalTransforms}
\end{table}

The source transformations also need to handle JavaScript syntax where literals cannot simply be wrapped into a function call.
Special handling is necessary for function declarations and accessor functions.


\subsubsection{Function Declarations}

Function declarations are named functions, where the name is available in the function itself and also in the surrounding scope.
% TODO: code listings that show the differences..

% TODO: show with why declarations cannot be just wrapped.. functions passed as arguments are always expressions..

The function declaration are transformed to function expressions that are assigned to matching variable names.
% TODO: show transformation with a table 
This way, \lstinline|function funcName(..) {..}| becomes \lstinline|var funcName = function funcName(..) {..}|.

In addition, because function declarations get hoisted in JavaScript, the source transformation moves the transformed function declarations to the beginning of the defining scope while preserving the original order among the transformed function declarations.


\subsubsection{Accessor Functions}

Accessor functions are functions that are executed transparently instead of ordinary reading or writing a property of an object.
The following example snippet shows the definition of a \lstinline{person} object, where two accessor functions allow to just read and write \lstinline{person.age} while the value actually is stored under a slightly different name:

% TODO: captions, \refs..

\iffalse
\begin{verbatim}\fi
\begin{code}[lst:example]{}{float,numbers=left}
var person = {
    __age: 0,
    get age() {
        return this.__age;
    },
    set age(val) {
        this.__age = val;
    }
}
\end{code}
\iffalse
\end{verbatim}\fi

Wrapping the two accessor functions in-place into \lstinline{proxyFor} functions would not yield valid JavaScript code.
Therefore, the source transformations create the objects first without accessor functions and explicitly define the accessor functions using \lstinline{Object.defineProperty} afterwards.
These two steps are done in an anonymous functions that is applied directly.
This allows to have the newly created object available in a temporary variable for the \lstinline{Object.defineProperty}-function call without having this temporary variable pollute the originally surrounding scope's variable bindings.
That is, the presented example with accessor functions would be transformed to the following code: \\

\iffalse
\begin{verbatim}\fi
\begin{code}[lst:example]{}{float,numbers=left}
var person = function() {
    var newObject = lively.proxyFor({
        __age: 0;
    });
    Object.defineProperty(newObject, "age", {
        get: lively.proxyFor(function age() {
            return this.__age;
        }),
        set: lively.proxyFor(function age(val) {
            this.__age = val;
        }),
        enumerable: true,
        configurable: true
    });
    return newObject;
}();'
\end{code}
\iffalse
\end{verbatim}\fi





\subsection{Wrapping Built-in Functions}

Some functions are not created from literals, but are built into the JavaScript engines.
These built-in functions are, therefore, not proxied by transforming literals.
For example, the built-in data types like objects and arrays can be created by evaluating \lstinline{new Object()} and \lstinline{new Array()} as well as \lstinline{Object()} and \lstinline{Array()}.
We transform these built-in constructor functions explicity, wrapping each into the proxying function.
Further, the proxy's \emph{construct}- and \emph{apply}-traps then also return proxies.

Besides built-in constructors, there are also very specific built-in functions, that we transform separately to specific alternatives.
One example for such functions is the global \lstinline{eval()} function.
\lstinline{eval} takes arbitrary code which might express an arbitrary object structure and which, therefore, might require proxies for multiple objects.
In the specific case of eval, we proxy \lstinline{eval}'s result but additionally also pass its string argument through the source transformations.

Similar to the wrapping of object literals without accessor functions, of arrays, and of function expressions, the source transformation also wraps specific built-in functions into calls to the \lstinline{proxyFor()} function.
For example, \lstinline{Array(100)} becomes \lstinline{proxyFor(Array(100))}.
As every call to such built-in objects is transformed in this way, the same Object---in this case the \lstinline{Array} function---gets passed to the \lstinline{proxyFor} function multiple times.
To nevertheless return the same proxy, we use a map that associates objects with their proxies.
This \emph{Proxy Table} is a weak-key map and does prevent the version objects used as keys from being gargabe collected when their proxies get garbage collected.
Using the same proxies for objects is also necessary for identity checks.
That is, the following statement yields true for an arbitrary object \lstinline{obj}: \lstinline{proxyFor(obj) === proxyFor(obj)}.


% This is also true for the \emph{apply} trap, because, for example, the \lstinline{Array} and \lstinline{Object} functions also create objects when called without the \lstinline{new} operator and, for example, the built-in function \lstinline{Object.create(proto)} also returns a new object.

% In fact, with our proxy handler the \emph{get}-trap also makes sure to return always proxy in this fashion.

% not just the apply-traps, but also the get-traps return ensuredProxied:
% while all application-specific objects properties are expected to be proxied, some built-in objects in JavaScript refer to their properties directly and not through proxies.
% These properties are required to be proxied as well to handle proxies correctly.

% this way. it is enough to wrap Object into proxyFor.. to have Object.create be proxied as well
% So, wrapping \lstinline{Object} is enough to have \lstinline{Object.create(proto)} return a proxy: \lstinline{Object.create(proto)} transforms to \lstinline{proxyFor(Object).create(proto)} and getting the property \emph{create} then returns a proxy for the function, which when applied also returns a proxy.
% so the source transformations wrap a number global objects, but do not wrap each of the properties.




% The built-in functions could be overwritten globally to return proxies for the new objects, but our implementation of object versioning is itself just a JavaScript library and makes itself use of the built-in data types.
% Additionally, at the time of writing, some JavaScript engines do not allow to overwrite particular built-in functions and we want our implementation of object versioning to work in every JavaScript engine that supports the ECMAScript 6 Direct Proxies.




\subsection{Returning Proxies from Constructor Functions} 

In JavaScript, all functions can be used as constructors and create new objects when called with the \emph{new} operator.

% TODO: code example: function defintion / literal function + function applied with \lstinline{new} => new object

% function object created from function literal or built-in. either way, function object is proxied through STEPS_ABOVE.
% construct trap used to return the new object proxied. trap-code: return this.ensureProxied(..).


% Functions expressed as function literals get proxied with the previous source transformations.
% Proxies for functions then return proxies for the newly created objects from their \emph{construct}-traps.

% % The last line of code of the \lstinline{construct} function of our proxy handler is therefore: \lstinline{return this.ensureProxied(newInstance)}.\\

% The result is passed to a \lstinline{ensureProxied()} function.












\subsubsection{Implementation of Source Transformations}

Our implementation uses \emph{UglifyJS}\footnote{\url{http://github.com/mishoo/UglifyJS2}, accessed March 12, 2014} for all source transformations.
UglifyJS parses source code without relying on JavaScript exceptions.
Therefore, when code is transformed before evaluation and the transformation steps do not yield exceptions that could be caught by an open debugger.
In addition, UglifyJS supports Source Maps \footnote{\url{https://docs.google.com/document/d/1U1RGAehQwRypUTovF1KRlpiOFze0b-_2gc6fAH0KY0k/edit\#heading=h.ue4jskhddao6}, accessed May 2, 2014}.
This way, the browser's developer tools, including the debugger, present the original sources, even though transformed code is executed.

% rough sketch of how transformations are done???
    % code of modules passed to source transformation. parsing, AST transformations, printing. 




\section{Workarounds for the Current State of the Proxies} \label{sec:IMPLEMENTATION:4}

% TODO: each subsection should be divided into a problem and a solution/approach/workaround part/subsubsection

Certain workarounds are required due to the preliminary status of the proxy implementation.

While the specification of Direct Proxies already progressed from proposal status to being part of the current ECMAScript 6 Draft\footnote{\url{http://people.mozilla.org/~jorendorff/es6-draft.html}, accessed March 5th, 2014}, it is still in draft status and not yet implemented in the JavaScript engines used by Chrome and Firefox.
These two engines currently both implement two different deprecated proposals for the proxies instead of the most recent ECMAScript 6 Draft.
The \emph{harmony-reflect} library\footnote{\url{http://github.com/tvcutsem/harmony-reflect}, accessed February 3, 2014, used version 0.0.11}, however, provides the current API of Direct Proxies for recent versions of both Chrome and Firefox, on top of the implementations of the different deprecated proposal states.
That is, our implementation uses the \emph{harmony-reflect} shim and, thereby, works in both Chrome and Firefox.

However, even with the shim, three issues need to be addressed with technical workarounds:

\begin{enumerate}
    \item The proxies have to be provided with an actual target, even when implementing virtual objects, and consistency invariants compare return values of traps to the state of the actual target. 
    \item The proxies do not intercept the \lstinline{instanceof} operator, but always delegate to the current target.
    \item Certain built-in JavaScript functions dont handle proxies correctly.
\end{enumerate}

These workarounds might no longer be necessary once the ECMAScript 6 standard gets finalized and fully implemented by the JavaScript engines. 


\subsection{Disabling Target Object Invariants}

% While these are supposed to stand-in for particular objects, their \emph{targets}, they can also be used as fully virtual objects.

\todo{add something on \url{http://wiki.ecmascript.org/doku.php?id=harmony:direct\_proxies\#virtual_objects}}
% when using the objects as virtual objects as described in in the ECMAScript wiki's documentation... still the invariants..

Using the \emph{harmony-reflect} shim in a recent Chrome, proxies require an actual \emph{target} object even though the current ECMAScript 6 draft says otherwise\footnote{\url{http://people.mozilla.org/~jorendorff/es6-draft.html\#sec-proxy-object-internal-methods-and-internal-slots}, accessed April 15, 2014}.

When an actual target object is provided on creation of a proxy, the internal link between proxy and target object is fixed and cannot be changed at runtime.
Therefore, in our use case of the proxies, in which proxies stand-in for a group of objects from which one is chosen dynamically, the proxy is still linked to a particular object by default.
Additionally, the Direct Proxies are designed to ensure invariants between the trap behavior and the target's properties~\cite{Cutsem2013TRP}.

For example, when an object's properties are made immutable using \lstinline{Object.freeze}, the invariants ensure that the target object has in-fact been frozen, even if the trap delegates the operation to another object such as one of our versions of an object.
Further, another invariant ensures that all immutable values of the target object are returned correctly by the \lstinline{get}-trap, even when the \emph{get}-trap reads from another object than the target. 
That is, in our case, configuring any property as immutable would effectively declare make that property immutable for all versions.
Additionally, in this scenario, reading the property in a version where the value of this property is different from what it is in the immutable target would raise incosistency errors.

For this reason, we adapted our copy of the \emph{harmony-reflect} library.
In particular, we added a boolean flag to the Proxy constructor function that indicates whether a proxy is standing in for one target object or is a virtual object.
This boolean flag is then used to disable all consistency checks when a proxy is meant to be a virtual object.

The proxy constructor still requires an actual object as \emph{target} object, but this target object is now only used for the \lstinline{typeof} operator for the virtual object-kind of proxies.

The \lstinline{typeof} operator returns only a distinction between a certain set of types in JavaScript.
The only necessary difference for the \lstinline{typeof} operator is that a proxy representing the versions of an object or array has an object as target, while a proxy representing the versions of a function needs to have a function as target.
This difference in the type of supplied target objects is also necessary as the \emph{apply}- and \emph{construct}-trap are also only called on proxies with functions as targets.


\subsection{Forwarding the \emph{Instanceof} Operator}

There is currently no \lstinline{instanceof} trap to intercept the behavior of the \lstinline{instanceof} operator.

The \lstinline{instanceof} operator returns the of an object, which is defined by an object's prototype chain in JavaScript.
The prototype of an object is a property and can be changed at runtime.
It gets versioned as any other property.
The \lstinline{instanceof} operator, thus, also needs to be delegated to the current version of an object.
% When creating a proxy a \emph{target} object can be provided to be linked from the proxy and the \lstinline{instanceof} operator would follow this link to report the target's type, but as the target link cannot be changed later on, while proxies stand-in for multiple versions with potentially different types, this is not an option.

For this, we implemented a custom \lstinline{Object.instanceof()} function, which implements the semantics of the \lstinline{instanceof} operator but does delegate to the correct version of an object.
All usage of the \lstinline{instanceof} operator is then transformed to usage of the custom function.
These transformations change code as shown by the following example: 

\iffalse
\begin{verbatim}\fi
\begin{code}[]{}{float}
anObject instanceof Type -> Object.instanceof(anObject, Type)
\end{code}
\iffalse
\end{verbatim}\fi

While there is no trap for the \lstinline{instanceof} operator in the current specification, trapping the operator is under discussion\footnote{\url{http://wiki.ecmascript.org/doku.php?id=harmony:direct_proxies\#discussed_during_tc39_july_2012_meeting_microsoft_redmond}, accessed May 1, 2014}.


\subsection{Unwrapping Versions for Native Code} \label{subsec:IMPLEMENTATION:4.3}

Some built-in JavaScript functions react to proxies with actual errors, return wrong results, or otherwise do not provide the usual functionality.
Therefore, these functions need to be provided with actual objects instead of the proxy.
% This is unproblematic for most cases as, given JavaScript's single-threaded and cooperative execution in browser engines, which version of an object is to be used is not expected to change during the execution of a built-in function.
Our implementation provides the current version of an object to these built-in functions through proxy handler traps and patched built-in functions.

The \emph{apply}-trap, called when proxied functions get applied, unwraps all arguments and the \emph{thisContext} object, when the applied function is a built-in functions that does not handle proxies correctly.
This is for example the case for the functions that provide access to the browser's \ac{dom}.

The \emph{set}-trap sometimes needs to unwrap the assigned value when that value is proxied and gets assigned to slots that cannot handle proxies.
This is for example the case with the \lstinline{onreadystatechange} slot of \emph{XMLHttpRequest} objects.
This particular slot holds functions as callbacks of asynchronous HTTP requests, which, however, do not get called when these functions are proxies.

Unwrapping a particular version from a proxy for an asynchronous callback is potentially problematic.
Though JavaScript does get executed with a single thread using cooperative scheduling by all popular browsers, other concurrent scripts might get executed and switch the global version before the callback can be called.
It is, however, unlikely that a callback function's \emph{properties} are changed while waiting for the response and even bad programming style as it introduces dependencies on network timing even without proxies.
Therefore, our implementation currently does not provide a workaround for this specific case, even though we are aware of this potential problem. 

In addition to unwrapping in traps, there are also certain immutable types that do not get proxied at all in our implemention such as strings.
Strings, however, do have methods and these also do not handle proxies correctly.
As the arguments to these methods is not handled in traps, because strings are not proxied, our implementation patches such functions with functions that unwrap proxies before executing the original functions.





\section{Limitations of the Implementation}

The current implementation has three limitations.

\paragraph{Availability of ECMAScript 6 Proxies}
Our implementation requires the ECMAScript 6 Direct Proxies to be available.
The proxies are, however, part of the next version of ECMAScript, which has currently neither been finalized nor completely implemented, as described in \ref{sec:IMPLEMENTATION:4}.
For this reason, our implementation works only in recent versions of Firefox and Chrome that already implement preliminary versions of the proxies.
For Chrome, users also need to enable the proxies through explicitly setting the \emph{Experimental JavaScript}-option.

\paragraph{Proxies Impede Developer Tools}
The current implementation of proxies impedes debugging.
The proxies are partly implemented by a JavaScript library and every trapped object access is visible in multiple frames in the debugger.
That is, the stack of the debugger is cluttered with frames that belong to the proxy implementation, not to the application source code.
In addition, the developer tools in Chrome do not handle proxies correctly under all circumstances.
In particular, hovering variable names that are currently bound to proxies yields errors and stepping into proxied functions does not work.

\paragraph{Concurrent JavaScript}
Even though JavaScript is executed with a single thread, scripts can be executed concurrently.
In general, scripts can be started from events and explicitly from other scripts using the \lstinline{setTimeout()} or the \lstinline{setInterval()} function.
Switching the system version can be problematic for such concurrently running scripts.
For example, the Lively Kernel makes use of the \lstinline{setInterval()} function to repeatedly execute a method of an object.
Re-establishing a previous version can interfere with such ticking behavior as the object or its method can be unavailable in previous versions.
For this reason, we stop scripts from future versions, when the system version changes.
We did, however, not succeed in finding a way to restart the scripts again when the future versions are re-established.
Thus, subsequentely undoing and redoing changes can stop concurrently running scripts.

% single-threaded and cooperative scheduling, but setTimeout / setInterval... stop asynchronous actions from following versions when undoing... source transformations, which scripts belongs to which version


% Even with these limitations our implementation is functional and integrated with the Lively Kernel.
% It allows programmers to undo and redo changes and is publicly available\footnote{\url{https://github.com/LivelyKernel/LivelyKernel/tree/object-versioning}}.
