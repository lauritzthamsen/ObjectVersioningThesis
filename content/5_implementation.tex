\chapter{Implementation with Proxies} \label{chapter:IMPLEMENTATION}

% additional TODO: .call + .apply, handle special cases also when functions are arguments these native functions..
% and TODO: [what about all other source transformations..??]

We implemented version-aware references for our approach to object versioning in JavaScript using proxies and source transformations.
This way, our prototype does not require changes to JavaScript engines, but only a certain JavaScript language feature.
In particular, it requires the Direct Proxies\footnote{\url{http://wiki.ecmascript.org/doku.php?id=harmony:direct_proxies}, accessed February 3rd, 2014} as proposed with Version 6 of ECMAScript, the standard that JavaScript follows.
These proxies can implement specific behavior to handle various kinds of access to them.
In our implementation, these proxies are used to delegate all access to a particular version of the object they stand in for.
To have these proxies intercept access to all mutable objects for which versions should be preserved, our prototype uses a combination of source transformations and proxy behavior.
In particular, the moment such mutable objects get created, our proxies get created for the new objects and references to these proxies are returned instead of references to the actual objects.
The proxies then preserve and choose versions of their object corresponding to a global version identifier.
This version identifier effectively declares one particular state of the programming runtime, consisting of those versions of objects that are associated with that runtime state, to be read and written.
Therefore, to change the entire runtime state as, for example, an undo and redo would require only the global version identifier needs to be changed.


\section{Using ECMAScript 6 Direct Proxies}

Our implementation of version-aware references is based on the ECMAScript 6 Direct Proxies.
While these are supposed to stand-in for particular objects, their \emph{targets}, they can also be used as fully virtual objects.
In our case, the proxies represent multiple different versions of an object and do not delegate to any of those by default.
Instead, they choose one of potentially many versions of an object to delegate to dynamically.
Further, the Direct Proxies in our implementation also stand-in transparently for the multiple versions of an object.
That is, the proxies always do behave---except for some specific introspection cases---as if they were a specific version of an object and programmers should not have to be aware about the versioning proxies.

The behavior of the proxies can be specified in \emph{Traps}, methods to specify how exactly specific access to the proxy should be handled, which are bundled in a \emph{Handler} object that is supplied when creating the proxies.
Our implementation of version-aware references using proxies is, thus, defined by our specific proxy handlers that provide the necessary delegation behavior through the traps.
Specific traps are called for specific access. 
For example, when \lstinline{person} is a proxy and \lstinline{person.age} is called, the \emph{get}-trap is called and receives \lstinline{target}, \lstinline{name}, and \lstinline{receiver} as arguments.
In our implementation, the \emph{target} parameter is irrelevant as we use proxies as virtual objects, not to be associated with any particular target object.
The other two parameters, however, are relevant and, for the given example, the \emph{name} would be the string \'age\' and \emph{receiver} would be the proxy for \emph{person} again.
Our proxy handler's \emph{get}-trap is implemented as indicated by the following code snippet, which is, however, simplified in that it leaves out much of the handling of special cases that are due to the shortcomings of the used implementation of Direct Proxies:

\iffalse
\begin{verbatim}\fi
\begin{code}{}{}
get: function(dummyTarget, name, receiver) {

    var version = this.currentVersion();
    
    // proxy meta information and other special cases..
    if (name === 'isProxy') {
    // ...
    if (name === 'proxyTarget') {
    // ...
    
    result = version[name];
    
    return this.ensureProxied(result);
}
\end{code}
\iffalse
\end{verbatim}\fi

First, the current version gets determined using the \lstinline{currentVersion()} function, which is explained in \ref{sec:IMPLEMENTATION:3}.
Then, multiple special cases are handled, including special properties to allow to determine whether a proxy, which otherwise behaves transparently, actually is a proxy or to allow to unwrap the current version from the proxy.
In the simple scenario that we are only reading a property named \lstinline{age} from an object \lstinline{person} all these special conditions fall through.
Instead, the trap reads the property from the current version and returns the result.
The result is passed to the \lstinline{ensureProxied()} function to make sure that mutable objects are proxied.
When a mutable object that is not wrapped in a proxy is passed to this function, the function returns a proxy instead.
This is necessary as, while application-specific objects properties are expected to be proxied, some built-in objects in JavaScript engines refer to their properties directly and not through proxies, but these properties might be required to be proxied to be versioned and to implement version-aware behavior.

Besides the \emph{get}-trap, our proxy handler similarly provides behavior for all other traps, which also delegate to the currently selected version of the object.
The following table \todo{table caption and refer its ref number} provides an overview over the implemented proxy traps:

\begin{table}[h]
\begin{tabular}{|l|l|r|}
\hline
get: function(target, name, receiver) \\ \hline
set: function(target, name, value, receiver) \\ \hline
apply: function(target, suppliedThisArg, suppliedArgs) \\ \hline
construct: function(target, args) \\ \hline
has: function(target, name) \\ \hline
hasOwn: function(target, name) \\ \hline
defineProperty: function(target, name, desc) \\ \hline
deleteProperty: function(target, name) \\ \hline
getOwnPropertyNames: function(target) \\ \hline
getPrototypeOf: function(target) \\ \hline
freeze: function(target) \\ \hline
seal: function(target) \\ \hline
preventExtensions: function(target) \\ \hline
isFrozen: function(target) \\ \hline
isSealed: function(target) \\ \hline
isExtensible: function(target) \\ \hline
enumerate: function(target) \\ \hline
keys: function(target) \\ \hline
\end{tabular}
\end{table}

All these traps fire either when the proxy is accessed with specific JavaScript operators or when it is passed to meta-programming facilities provided by built-in globals\footnote{\url{http://wiki.ecmascript.org/doku.php?id=harmony:direct_proxies\#api}, accessed March 5, 2014}.
For example, the \lstinline{preventExtensions}-trap fires when a proxy is passed to the built-in \lstinline{Object.preventExtensions()} function, which prevents new properties from subsequentely being added to an object.

There currently is no \lstinline{instanceof} trap to intercept the behavior of the \lstinline{instanceof} operator since the Direct Proxies currently do not provide such a trap.
However, the type of an object in JavaScript is defined by an object's prototype chain, which may change at runtime.
So, while the prototype of an object is a property and gets versioned as well, the \lstinline{instanceof} operator does not get trapped and, thus, is not delegated to the current version of the object.
Therefore, this particular object access needs to be handled by a custom \lstinline{Object.instanceof()} function, which implments the semantics of JacaScript's \lstinline{instanceof} operator but does delegate the necessary checks when a versioning proxy gets passed to it.
To exchange all usage of the \lstinline{instanceof} operator with application of this custom function, we transform JavaScript sources before execution.
These transformations change, for example, \lstinline{anObject instanceof SomeConstructor} to \lstinline{Object.instanceOf(anObject, SomeConstructor)}.

The handler object is an ordinary JavaScript object and, thus, can hold arbitrary properties.
Besides providing the trap methods, our proxy handler objects also hold the versions of the represented object.
For this, the proxy handler uses an object as dictionary that associates version identifiers with the different versions of the object.
This version dictionary is used by the handler's \lstinline{currentVersion} function that decides which of the versions should currently be accessed.
Having the handler object hold all the versions of conceptionally one object also takes care of garbage collection.
When a proxy gets garbage collected, all versions get garbage collected as well.

Certain workarounds are required to make our proxy-based implementation of version-aware references work, due to the preliminary status of the proxy implementation.
While the specification of Direct Proxies already progressed from proposal status to being part of the current ECMAScript 6 Draft\footnote{\url{http://people.mozilla.org/~jorendorff/es6-draft.html}, accessed March 5th, 2014}, it is still in draft status and not yet implemented in the JavaScript engines used by the Chrome browser and the Firefox browser.
These two engines currently both implement two different deprecated proposals for the proxies instead of the most recent ECMAScript 6 Draft.
The \emph{harmony-reflect} library\footnote{\url{http://github.com/tvcutsem/harmony-reflect}, accessed February 3, 2014, used version 0.0.11}, however, provides the current API of Direct Proxies for recent versions of both Chrome and Firefox, on top of the implementations of the different deprecated proposal states.
That is, our versioning proxies prinicipally work in both Chrome and Firefox and rely on the \emph{harmony-reflect} shim for this.
However, even with the shim, three issues need to be addressed with technical workarounds that might no longer be necessary once ECMAScript 6 standard gets finalized and fully implemented by the JavaScript engines: 

\begin{enumerate}
    \item Certain built-in JavaScript functions dont handle proxies correctly.
    \item The proxies have to be provided with an actual target, even when implementing virtual objects and consistency checks compare return values of traps to the state of the actual target. 
    \item It is not possible to set the prototype of an object when the current prototype is a proxy due to a bug in Chrome.
\end{enumerate}

Some built-in functions react to proxies with actual errors, return wrong results, or otherwise do not provide the usual functionality.
Therefore, these functions need to be provided with actual objects instead of the proxy.
This is unproblematic for most cases as, given JavaScript's single-threaded and cooperative execution in browser engines, which version of an object is to be used is not expected to change during the execution of a built-in function.
Our implementation provides the current version of an object to these built-in functions through proxy handler traps and patched built-in functions.
The \emph{apply}-trap, called when proxied functions get applied, unwraps all arguments and the \emph{thisContext} object, when the applied function is a built-in functions that does not handle proxies.
This is for example the case for the functions that provide access to the browser's document model.
Further, the \emph{set}-trap sometimes needs to unwrap the assigned value when that value is proxied and gets assigned to slots that cannot handle proxies.
This is for example the case with the \lstinline{onreadystatechange} slot of \emph{XMLHttpRequest} objects.
This particular slot holds functions as callbacks of asynchronous HTTP requests, which, however, do not get called when these functions are proxies.
This is particular case is potentially problematic.
Though JavaScript does get executed with a single thread  using cooperative scheduling by all popular browsers, other concurrent scripts might get executed before a script ends whenever a script is waiting on I/O like, for example, an asynchronous request to finish.
As this---changing a callback function or its properties while waiting on the response---is unlikely, bad programming style as it introduces dependencies to the timing of network I/O, and is through network communication already at the limit of what object versioning can due in general, our implementation currently does not provide a workaround for this specific case, even though we are aware of this potential problem. 
In addition to these behavior implemented in traps, there are also certain immutable types that do not get proxied at all in our implemention as, for example, strings.
However, some methods of, for example, string objects also do not handle proxies correctly and our implementation, therefore, patches those functions to also unwrap proxies before executing the original functions.


\todo{dummytarget: rework the following part..} 
% TODO: dummytarget.. say something about our adaptions to harmony-reflect: no-target flag.. to disable consistency checks of configured properties (frozen..etc)
Instead, as the implementation of the proxies currently requires one particular \emph{target} object even for proxies intended to be virtual objects, the \lstinline{instanceof} operator returns the type of that initial target object.



\todo{Second, proxies as prototypes (relation cant be changed.. traps of proto fire when inheriting object is accessed.. problem is Chrome and only the proto case...)... protoProxy property on objects..}
\todo{maybe explain the design decisions behind the protoID}
% firing proto traps, except when using the new Reflect API, but can't use Reflect.setPrototype in recent Chromes, where Lively runs best..





\section{Returning Proxies For New Objects}

As we use proxies as version-aware references while the JavaScript runtime still only provides ordinary references, references that usually would point to objects directly now need to point to proxies instead. 
References to the same object further need to point to the same proxy for that object.
This is necessary because identity checks will compare proxies instead of objects.
Also because, at least in our implementation, proxies actually hold the different versions of their objects as opposed to looking the objects up in a separate data structure, one particular proxy knows the versions of an object.
Therefore, all references that would normally refer to the object need to refer to the proxy that stands-in for the versions of the objects.
To achieve this, we create proxies immediately for new objects and return the proxies instead of the objects, so a reference to the proxy gets passed around instead of the reference to the actual object, which instead is captured in the versions dictionary of the proxy.

We need to create proxies for all new mutable state, which in JavaScript comprises of objects, arrays, and functions.
JavaScript allows to create these kinds of objects through literal expressions, through constructors, and through built-in functions.
That is, functions are objects and can also have properties in JavaScript.
Constructors can be built-in as, for example, \lstinline{Array} or literal functions.
Both create new objects when applied with the \lstinline{new} keyword, however built-in constructors also return new objects when applied without the \lstinline{new} operator.
Other built-in functions are, for example, \lstinline{eval} or \lstinline{Object.create(..)}.
To actually create and return proxies from all these three different ways to create objects, our implementation transforms sources before execution.
The built-in functions could be overwritten globally to return proxies for the new objects, but our implementation of object versioning also uses these built-in data types internally.
Additionally, at the time of writing, some JavaScript engines do not allow to overwrite particular built-in functions, while we want our implementation of object versioning to work in every JavaScript engine that supports ECMAScript 6 Direct Proxies.
Therefore, we transform specific literal expressions and specific built-in functions, while custom constructors return proxies through particular proxy trap behavior.

The source transformations for introducing our proxy-based version-aware references wrap the original expressions into the \lstinline{proxyFor} function, which returns a proxy for an object.
The transformations of literal objects, arrays, and functions are shown by example in following table.

\begin{center}
    \begin{tabular}{| l | l | l | l |}
    \hline
    Type & Original code & Transformed code \\ \hline
    \emph{Objects} & \lstinline|{name: 'James', age: 24}| & \lstinline|proxyFor(name: 'James', age: 24)| \\ \hline
    \emph{Arrays} & \lstinline|[person1, person2]| & \lstinline|proxyFor([person1, person2])| \\ \hline
    \emph{Functions} & \lstinline|function (a, b) {..}| & \lstinline|proxyFor(function (a, b) {..})| \\ \hline
    \end{tabular}
\end{center}

Besides straightforward wrapping of literal objects, literal arrays, and function expressions, the source transformations also needs to handle function declarations and accessor functions.
Function declarations are named functions, where the name is available for recursion in the function itself, but also makes the function accessible by the name in the surrounding scope.
For this reason, the function declaration get transformed to function expressions that are assigned to matching variable names.
This way, \lstinline|function funcName(..) {..}| becomes \lstinline|var funcName = function funcName(..) {..}|.
In addition, because function declarations get hoisted in JavaScript, the source transformation moves the results of transforming function declarations to the beginning of the defining scope, while the order among function declaration is preserved.
Accessor functions are functions that are executed instead of ordinary read or write access to properties.
The following example snippet shows the definition of a \lstinline{person} object, where two accessor functions allow reading and writing \lstinline{person.age} while the value is stored under a different name: \\
\iffalse
\begin{verbatim}\fi
\begin{code}{}{}
var person = {
    __age: 0,
    get age() {
        return this.__age;
    },
    set age(val) {
        this.__age = val;
    }
}
\end{code}
\iffalse
\end{verbatim}\fi

Wrapping the two accessor functions in-place into \lstinline{proxyFor} functions does not yield valid JavaScript code.
Therefore, the source transformations wrap object literals with accessor functions into anonymous functions that are directly applied, create the objects first without accessor functions, and explicitly define the accessor functions using \lstinline{Object.defineProperty}.
That is, the presented example with accessor functions would be transformed to the following code: \\

\iffalse
\begin{verbatim}\fi
\begin{code}{}{}
var person = function() {
    var newObject = lively.proxyFor({
        __age: 0;
    });
    Object.defineProperty(newObject, "age", {
        get: lively.proxyFor(function age() {
            return this.__age;
        }),
        set: lively.proxyFor(function age(val) {
            this.__age = val;
        }),
        enumerable: true,
        configurable: true
    });
    return newObject;
}();'
\end{code}
\iffalse
\end{verbatim}\fi

Similar to the wrapping of object literals without accessor functions, arrays, and function expressions, the source transformation also wraps specific built-in functions into calls to the \lstinline{proxyFor()} functions.
For example, \lstinline{Array(100)} becomes \lstinline{proxyFor(Array(100))}.
As every call to such built-in objects is transformed in this way, the same Object---in this case the \lstinline{Array} function---gets passed to the \lstinline{proxyFor} function multiple times.
To nevertheless return the same proxy and, therefore, the same identity, we use a map that associates objects with their proxies.
This \emph{Proxy Table} is a weak-key map and does not prevent neither objects nor proxies from being gargabe collected.
Therefore, the following statement returns true for an arbitrary object \lstinline{obj}: \lstinline{proxyFor(obj) === proxyFor(obj)}.

When objects get created through applying functions with the \lstinline{new} operator, the new objects also need to be proxied.
However, as the functions are already proxied through source transformations of function literals and the built-in \lstinline{Function} constructor, our implementation does not transform the applications of the \lstinline{new} operator, but uses the \emph{construct} trap to return proxies from constructor applications.
The last line of code of the \lstinline{construct} function of our proxy handler is therefore: \lstinline{return this.ensureProxied(newInstance)}.
This is also true for the \emph{apply} trap, because, for example, the \lstinline{Array} and \lstinline{Object} functions also create objects when called without the \lstinline{new} operator and, for example, the built-in function \lstinline{Object.create(proto)} also returns a new object.
In fact, with our proxy handler the \emph{get}-trap also makes sure to return always proxy in this fashion.
So, wrapping \lstinline{Object} is enough to have \lstinline{Object.create(proto)} return a proxy: \lstinline{Object} transforms to \lstinline{proxyFor(Object)}, while getting the property returns a proxy for the \emph{create} function, and applying that in turn also returns a proxy.

The implementation of source transformations uses \emph{UglifyJS}\footnote{\url{http://github.com/mishoo/UglifyJS2}, accessed March 12, 2014} for parsing and AST transformations.
UglifyJS parses without using JavaScript exceptions for backtracking and, therefore, does not interfere with debugging.
In addition, UglifyJS also supports Source Maps \todo{footnote with url to Source Maps necessary..}, which allow to view and debug the original code with the browser's developer tools, even though the transformed code is executed.



\section{Providing Simple Global Versions for Undo and Redo} \label{sec:IMPLEMENTATION:3}

JS execution: single-threaded, cooperative scheduling 

\begin{itemize}
    \item global versions, linear versioning, data structure is an version object with a predecessor and a successor
    \item copy-on-write for versions of objects (set-trap, apply-trap for some array methods), shallow copy
\end{itemize}


However, some additional logic is required to update state that is not part of the runtime state as, for example, the HTML document presented by browsers.


The version also may have a predecessor and a successor.
This way, for undo and redo the current version of the runtime only has to be set on the predecessor or the successor.
Preserving the predecessor also allows to preserve versions of objects only on writes as, when objects have not changed in versions, the first available previous version of the object can be read.

% Besides this copy-on-write optimization, another optimization would be to not copy objects completely, but to only store changes.
% this could, for example, also apply prototypical inheritance to have versions that delegate to previous versions and only express the differences between versions.

The association proxies associate all available versions of an object with a version of the runtime.


When a proxy is created for an object, the object is preserved as the version of the object that belongs to the current version of the runtime.
On writes, the proxies copy the latest version of the object and then do the write on the copy instead.
The copy gets marked with the identifier of the new version of the runtime.
When the proxies now choose a particular version of the object they stand-in for, they all use the version associated with the current version of the runtime or of a previous versions.


\begin{itemize}
    \item source transformations enabled right after the object versioning code, i.e. the implementation of the proxies and the source transformation, which are both themselves implemented using ordinary javascript code, gets loaded. by patching the eval that is used for evaluated modules that get loaded with an eval(transformSource.. so modules are not versioned, but classes / layers / traits are.. e.g. a class is versioned, so in different versions there might be different class-side state... so, for example, modules are currently in our implementation the roots of the version-aware reference graphs that prevent garbage collection, 
\end{itemize}


\begin{itemize}
    \item undo / redo implementation, example situation
    \item redraw of the world as DOM objects / DOM relations are not part of the object versioning in JavaScript: DOM redrawing on undo / redo: DOM is out of versioning scope, doesn’t use version-aware references, has to be changed / redrawn on undo/redo explicitly
    
\end{itemize}








\section{Limitations of the Implementation}




% \subsection{Limitations for Concurrent JavaScript}


single-threaded and cooperative scheduling, but setTimeout / setInterval... stop asynchronous actions from following versions when undoing... source transformations, which scripts belongs to which version


% \subsection{Limitations in the Execution Environment}

\begin{itemize}
    \item  the proxies are not fully implemented in all browsers, yet, and some built-in browser tools for developers also do not support them entirely.
    only works in recent Chromes, with experimental JavaScript / or Firefox (without adaption)... ECMAScript 6 Direct Proxies not yet fully implemented, not yet stable..
.. own adaptions to the shim library, to bridge the gap between the current version of the shim and information from the Spec draft (pass null for the target of a proxy)... fully virtual objects vs. consistency checks.. adapted library to fit spec draft (pass null as target of a fully virtual object, then no trap-return-value-consistency-with-target checks)
\end{itemize}
    

% \subsection{Usability: Impact on Debugging}

\begin{itemize}
    \item adds lots of low-level frames, dev tools don’t handle proxies well (hovering, labels, stepping into), but we have at least source maps, so developers can debug the original code and not the result of our source transformation
    \item debugging: interference with Chrome's developer tools, which don’t handle proxies well (hovering, labels, stepping into)
\end{itemize}


Besides impacting code execution, the current implementation of object versioning for Lively also  with certain programming tasks.
Though the version-aware references should be transparent to programmers, in the current proxy-based implementation proxies show during debugging.
However, the developer tools show the original source code, despite the implementation's transformations of the sources.

Nevertheless, our implementation is functional and integrated with Lively.
It allows programmers to undo and redo changes and is publicly available\footnote{\url{http://github.com/LivelyKernel/LivelyKernel/commit/858aca8c5ac0f203061823f463fda9b434131636}}.
