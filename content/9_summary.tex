\chapter{Summary} \label{chapter:SUMMARY}

% Approach
This work introduced an approach to preserving access to previous states of programming systems such as the Lively Kernel.
The approach is based on version-aware references.
These references manage different versions of objects transparently.
They resolve to one of multiple versions of objects and to which ones can easily be changed.
Thereby, different preserved states can be re-established.

We presented a design for our approach that uses proxies for version-aware references.
Instead of actually using alternative references, ordinary references refer to proxies and proxies forward all interactions transparently to the right versions.
The design allows implementing version-aware references without any adaptions to existing execution engines---neither for alternative references nor for garbage collection of versions.\\
For each new object that is created, a proxy is created and returned instead of the new object.
Thus, references to proxies are passed around and all access goes through the proxies.
Moreover, only the proxies refer to the versions of an object.
So, all versions of an object are reclaimed together with their proxy by the ordinary garbage collector.\\
Returning new proxies for new objects is achieved using source transformations.
Program sources are transformed when loaded and do not have to be adapted manually.

% Significance, Practical applications
We implemented our approach to object versioning in JavaScript.
It allows preserving and re-establishing versions of systems.
The implementation is optimized for fine-grained histories.\\
To switch the version of the system, only a global version identifier has to be changed.
Using this identifier, the proxies choose versions of objects dynamically.
This way, it is not necessary to re-configure all proxies for switching versions.\\
To preserve versions of the system, the proxies copy object versions on write: When a proxy intercepts a write operation in a \emph{system version} for which no \emph{object version} exists, it copies a previous \emph{object version}.
Until then, proxies reuse object versions from previous system versions.

We integrated our implementation into the Lively Kernel and made it work with our version-aware references.
Users can commit versions of the system state.
Using these versions, they can undo and redo changes.
This shows that our implementation works correctly in all situations tested.
The memory overhead is reasonable.
The execution overhead is, however, not yet practical: the execution of eight JavaScript benchmarks takes currently three orders of magnitude more time with our version-aware references.

% Outlook
In the future, the implementation could be improved by reducing the execution overhead.
In addition, the system should preserve relevant versions automatically and provide dedicated tools to developers.
